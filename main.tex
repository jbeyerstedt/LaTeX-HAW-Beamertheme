%
% Example for Beamer Presentations inspired by the 2017 Corporate Design
% at Hochschule für Angewandte Wissenschaften Hamburg
% - Author: Jannik Beyerstedt
% - License: CC BY-SA 4.0 (https://creativecommons.org/licenses/by-sa/4.0/)
% - some parts used from https://github.com/matze/mtheme
%
\documentclass[10pt]{beamer}

\usetheme{hawbeamer}
\usepackage{-settings/hawbeamerstyle}
\usepackage{appendixnumberbeamer}

\usepackage{booktabs}


% MANUAL HYPHENATION (whitespace separated items)
\hyphenation{}

% CUSTOM UNITS
\sisetup{locale = DE}


%--------------------------------------------------------------------
% DOCUMENT
%--------------------------------------------------------------------
\title[HAW Hamburg Beamer]{Titel der Präsentation}
\subtitle{Untertitel der Präsentation}
\author{Author Name}
\institute{Hochschule für Angewandte Wissenschaften Hamburg}
\date{30.11.2017}

\faculty{Absender (Fakultät oder Abteilung)}
\footerinfo{optionale Informationen}

\mode<presentation>
\begin{document}
%% ------------------------- TITLE PAGE -------------------------- %%
\maketitle

%% --------------------------- SLIDES ---------------------------- %%
% Uncomment these lines for an automatically generated outline.
\begin{frame}{Outline}
  \tableofcontents
  % \tableofcontents[hideallsubsections]
\end{frame}

\section{Introduction}
\begin{frame}[allowframebreaks]{Introduction}
  TODO: DAS MUSS ALLES IRGENDWIE BREITER

  Hier steht Lauftext. Begriffe können \emph{hervorgehoben werden}, der Beamer-Standard ist jedoch \alert{etwas wichtiges}.

  Mehr Text mit einer Aufzählung:
  \begin{itemize}
    \item Your introduction goes here!
    \item Use \texttt{itemize} to organize your main points.
  \end{itemize}
\end{frame}

\begin{frame}{Introduction2}
  \begin{block}{Default}
    Block content.
  \end{block}
  \begin{alertblock}{Alert}
    Block content.
  \end{alertblock}
  \begin{exampleblock}{Example}
    Block content.
  \end{exampleblock}
  \begin{example}
    \begin{itemize}
      \item 2 is prime (two divisors: 1 and 2).
      \item 3 is prime (two divisors: 1 and 3).
      \item 4 is not prime (\alert{three} divisors: 1, 2, and 4).
    \end{itemize}
  \end{example}
\end{frame}


\section{Some \LaTeX{} Examples}

\subsection{Tables and Figures}
\begin{frame}{Tables and Figures}
  \begin{itemize}
    \item Use \texttt{tabular} for basic tables --- see Table~\ref{tab:widgets}, for example.
    \item You can upload a figure (JPEG, PNG or PDF) using the files menu.
    \item To include it in your document, use the \texttt{includegraphics} command (see the comment below in the source code).
  \end{itemize}

  % Commands to include a figure:
  %\begin{figure}
  %\includegraphics[width=\textwidth]{your-figure's-file-name}
  %\caption{\label{fig:your-figure}Caption goes here.}
  %\end{figure}

  \begin{table}
    \centering
    \begin{tabular}{l|r}
    Item & Quantity \\\hline
    Widgets & 42 \\
    Gadgets & 13
    \end{tabular}
    \caption{\label{tab:widgets}An example table.}
  \end{table}
\end{frame}

\subsection{Mathematics}
\begin{frame}{Readable Mathematics}
  Let $X_1, X_2, \ldots, X_n$ be a sequence of independent and identically distributed random variables with $\text{E}[X_i] = \mu$ and $\text{Var}[X_i] = \sigma^2 < \infty$, and let
  $$S_n = \frac{X_1 + X_2 + \cdots + X_n}{n}
        = \frac{1}{n}\sum_{i}^{n} X_i$$
  denote their mean. Then as $n$ approaches infinity, the random variables $\sqrt{n}(S_n - \mu)$ converge in distribution to a normal $\mathcal{N}(0, \sigma^2)$.
\end{frame}

\section{References}
\begin{frame}{References}
  Some references to showcase [allowframebreaks] \cite{knuth92,ConcreteMath,Simpson,Er01,greenwade93}
\end{frame}

\appendix

\begin{frame}[fragile]{Backup slides}
  Sometimes, it is useful to add slides at the end of your presentation to
  refer to during audience questions.

  The best way to do this is to include the \verb|appendixnumberbeamer|
  package in your preamble and call \verb|\appendix| before your backup slides.
\end{frame}

\begin{frame}[allowframebreaks]{References}
  \bibliography{bibliography}
  \bibliographystyle{alpha}
\end{frame}

\end{document}
